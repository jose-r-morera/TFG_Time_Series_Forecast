
En este trabajo se ha desarrollado un sistema de predicción de variables meteorológicas, capaz de generalizar y realizar predicciones de ubicaciones no vistas. 
Se han estudiado diversas alternativas en el ámbito del aprendizaje profundo, y se ha observado que los modelos LSTM son los que mejores resultados brindan entre todos los modelos estudiados. Además,
se ha comprobado que estos resultados son buenos. Son mejores que los de modelos tradicionales para predicción de series temporales, y superan a los resultados de otras investigaciones 
recientes.

Se ha elaborado un flujo de trabajo completo, desde la obtención de datos, su tratamiento y limpieza, hasta la creación de los modelos de predicción. Se han 
realizado extensas pruebas de distintas teorías y configuraciones, con el fin de obtener los mejores resultados para este problema.

Uno de los hallazgos más relevantes es que, estudiando el tamaño de la ventana de datos pasados, se observa que empleando pocas mediciones, como máximo un día, se obtienen mejores predicciones que
con períodos más extensos. Esto es particularmente útil puesto que permite reducir el número de mediciones necesarias para realizar una predicción, haciendo los modelos 
más prácticos para su uso en la vida real.

Asimismo, se ha observado que el uso de un mayor número de puntos de medición en el conjunto de entrenamiento mejora la precisión de las predicciones.
En cambio, se ha descartado el uso de ruido sintético para aumentar dicho conjunto, al menos para las variables meteorológicas analizadas.

En cuanto a la arquitectura de los modelos, se concluye que las configuraciones más eficaces son relativamente simples, consistiendo en una capa LSTM bidireccional seguida de capas densas con pocas neuronas, sin necesidad de técnicas de regularización como dropout. 
Esta simplicidad facilita tanto el entrenamiento como la ejecución eficiente del modelo.

Por último, se ha desarrollado una aplicación web que pone al alcance de cualquier usuario realizar predicciones de forma sencilla. 
Esta aplicación permite al usuario seleccionar una estación de GRAFCAN y obtener la predicción de temperatura para las próximas horas.
Se trata de una prueba de concepto de la aplicabilidad del sistema desarrollado, y su potencial para ser utilizado en la vida real. La solución diseñada
es modular y escalable, siendo una propuesta seria para su uso en producción real.

\section{Líneas futuras}

Se ha constatado que el uso de un mayor número de puntos de medición en la construcción del conjunto de entrenamiento mejora los resultados de predicción.
La escalada inteligente de este conjunto, incluyendo información geográfica de las ubicaciones como la altitud y coordenadas, así como la construcción de una malla 
de puntos de medición podría constituir una línea de trabajo futura.

Por otra parte, se ha observado que la utilización de covariables adicionales, como la humedad o la presión atmosférica, en el caso de la temperatura, mejoran la precisión de los modelos.
El estudio de estas covariables y la inclusión de otras puede suponer una mejora en los resultados.

Finalmente, si bien se han estudiado diversos modelos de aprendizaje profundo, existen múltiples alternativas que no se han estudiado en este trabajo.
El estudio de modelos de última generación, como los transformers, supone otro ámbito de trabajo futuro.