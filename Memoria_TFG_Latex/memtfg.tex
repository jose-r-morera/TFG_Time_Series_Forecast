\documentclass[spanish,a4paper,12pt,oneside]{extreport}

\usepackage[dvips]{graphicx}
\usepackage[dvips]{epsfig}
\usepackage[utf8]{inputenc}
\usepackage[spanish]{babel}
\usepackage{alltt}

\usepackage[ruled,vlined,commentsnumbered,linesnumbered,inoutnumbered,titlenotnumbered,noend]{algorithm2e}
\SetKwRepeat{Do}{do}{while}

\usepackage{multirow}
\usepackage{array} 
\usepackage{amsfonts}
\usepackage{amsmath}
\usepackage{bigstrut}
\usepackage{booktabs}
\usepackage{caption}
\usepackage{chngpage}
\usepackage{float}
\usepackage{enumitem,lipsum}
\usepackage{graphicx}
\usepackage{lscape}
\usepackage{microtype}
\usepackage[numbers]{natbib}
\usepackage{pdflscape}
\usepackage{rotating}
\usepackage{subcaption}
\usepackage{ctable}
\usepackage{hyperref}
%\usepackage{enumerate} %deprecated
\usepackage{gensymb}
\usepackage{eurosym}
\usepackage{xcolor}
\usepackage{tabu}

\usepackage{lineno}
%\\linenumbers
%\setlength\linenumbersep{5pt}
%\renewcommand\linenumberfont{\normalfont\tiny\sffamily\color{gray}}

% TABLES

% END TABLES

\usepackage[top=2cm, bottom=2cm, left=2cm, right=2cm]{geometry}

\newenvironment{sourcecode}
{\begin{list}{}{\setlength{\leftmargin}{1em}}\item\scriptsize\bfseries}
{\end{list}}

\newenvironment{littlesourcecode}
{\begin{list}{}{\setlength{\leftmargin}{1em}}\item\tiny\bfseries}
{\end{list}}

\newenvironment{summary}
{\par\noindent\begin{center}\textbf{Abstract}\end{center}\begin{itshape}\par\noindent}
{\end{itshape}}

\newenvironment{keywords}
{\begin{list}{}{\setlength{\leftmargin}{1em}}\item[\hskip\labelsep \bfseries Keywords:]}
{\end{list}}

\newenvironment{palabrasClave}
{\begin{list}{}{\setlength{\leftmargin}{1em}}\item[\hskip\labelsep \bfseries Palabras clave:]}
{\end{list}}

\usepackage{bera}% optional: just to have a nice mono-spaced font
\usepackage{listings}
\usepackage{xcolor}

\colorlet{punct}{red!60!black}
\definecolor{background}{HTML}{EEEEEE}
\definecolor{delim}{RGB}{20,105,176}
\colorlet{numb}{magenta!60!black}

\lstdefinelanguage{json}{
    basicstyle=\normalfont\ttfamily,
    numbers=left,
    numberstyle=\scriptsize,
    stepnumber=1,
    numbersep=8pt,
    showstringspaces=false,
    breaklines=true,
    frame=lines,
    backgroundcolor=\color{background},
    literate=
     *{0}{{{\color{numb}0}}}{1}
      {1}{{{\color{numb}1}}}{1}
      {2}{{{\color{numb}2}}}{1}
      {3}{{{\color{numb}3}}}{1}
      {4}{{{\color{numb}4}}}{1}
      {5}{{{\color{numb}5}}}{1}
      {6}{{{\color{numb}6}}}{1}
      {7}{{{\color{numb}7}}}{1}
      {8}{{{\color{numb}8}}}{1}
      {9}{{{\color{numb}9}}}{1}
      {:}{{{\color{punct}{:}}}}{1}
      {,}{{{\color{punct}{,}}}}{1}
      {\{}{{{\color{delim}{\{}}}}{1}
      {\}}{{{\color{delim}{\}}}}}{1}
      {[}{{{\color{delim}{[}}}}{1}
      {]}{{{\color{delim}{]}}}}{1},
}

\begin{document}

\renewcommand\listtablename{Índice de Tablas}    
\renewcommand\listfigurename{Índice de Figuras}    

%%%%%%%%%%%%%%%%%%%%%%%%%%%%%%%%%%%%%%%%%%%%%%%%%%%%%%%%%%%%%%%%%%%%%%%%%%%%%%%
% First Page
%%%%%%%%%%%%%%%%%%%%%%%%%%%%%%%%%%%%%%%%%%%%%%%%%%%%%%%%%%%%%%%%%%%%%%%%%%%%%%%
\pagestyle{empty}
\thispagestyle{empty}


\newcommand{\HRule}{\rule{\linewidth}{1mm}}
\setlength{\parindent}{0mm}
\setlength{\parskip}{0mm}

\vspace*{\stretch{0.5}}

\begin{center}
\includegraphics[scale=0.8]{images/escuela-ingenieria-tecnologia-original}\\[10mm]
{\Huge Trabajo de Fin de Grado}
\end{center}

\HRule
\begin{flushright}
        {\Huge Predicción de series temporales meteorológicas mediante técnicas de aprendizaje profundo } \\[2.5mm]
        {\Large \textit{Weather time series forecasting using deep learning techniques}} \\[5mm]
        {\Large José Ramón Morera Campos} \\[5mm]


\end{flushright}
\HRule
\vspace*{\stretch{2}}
\begin{center}
  \Large La Laguna, \today
\end{center}

\setlength{\parindent}{5mm}

%%%%%%%%%%%%%%%%%%%%%%%%%%%%%%%%%%%%%%%%%%%%%%%%%%%%%%%%%%
% Signature page (add the official stamp)
%%%%%%%%%%%%%%%%%%%%%%%%%%%%%%%%%%%%%%%%%%%%%%%%%%%%%%%%%%
\newpage
\thispagestyle{empty}

D. {\bf Leopoldo Acosta Sánchez}, profesor Catedrático de Universidad adscrito al Departamento de Ingeniería Informática y de Sistemas de la Universidad de La Laguna, como tutor

\bigskip
D. {\bf Daniel Acosta Hernández}, investigador adscrito al Departamento de Ingeniería Informática y de Sistemas de la Universidad de La Laguna, como cotutor\pagestyle{empty}

\bigskip
\bigskip
{\bf C E R T I F I C A N}

\bigskip
\bigskip
Que la presente memoria titulada:

\bigskip
''{\it Predicción de series temporales mediante técnicas de aprendizaje profundo}''

\bigskip
\bigskip
\bigskip

\noindent ha sido realizada bajo su dirección por D. {\bf José Ramón Morera Campos}.

\bigskip
\bigskip

Y para que así conste, en cumplimiento de la legislación vigente y a los efectos
oportunos firman la presente en La Laguna a \today

%%%%%%%%%%%%%%%%%%%%%%%%%%%%%%%%%%%%%%%%%%%%%%%%%%%%%%%%%%
\newpage
\thispagestyle{empty}

{ \begin{LARGE}
Agradecimientos
\end{LARGE}

\hspace{3mm}

\begin{large}
Quisiera agradecer a todas las personas que han contribuido de forma directa o indirecta a la realización de este Trabajo de Fin de Grado. \\

En primer lugar, agradezco a mis tutores, Leopoldo Acosta y Daniel Acosta, por su orientación y disponibilidad a lo largo del desarrollo del trabajo. 
Su apoyo ha sido fundamental para alcanzar los objetivos propuestos. \\

También deseo expresar mi agradecimiento a los profesores y al personal de la Universidad por la formación recibida durante estos años, que se ve reflejada en este proyecto final. \\

Agradezco igualmente a mis compañeros y compañeras de grado por su colaboración y por el intercambio de ideas y experiencias que enriquecieron el proceso de aprendizaje. \\

Por último, quiero reconocer el apoyo de mi familia durante este periodo, así como de todas aquellas personas que, de una u otra forma, han hecho posible la finalización de esta etapa académica. \\
\end{large}

}
%%%%%%%%%%%%%%%%%%%%%%%%%%%%%%%%%%%%%%%%%%%%%%%%%%%%%%%%
\newpage
\thispagestyle{empty}

\bigskip
\begin{LARGE}
Licencia
\end{LARGE}

\bigskip
\bigskip
\bigskip
\bigskip

\begin{center}
\includegraphics[scale=1.8]{images/by-nc-sa_88x31}\\[5mm]
\end{center}

\begin{large}
© Esta obra está bajo una licencia de Creative Commons Reconocimiento-NoComercial-CompartirIgual 4.0 Internacional.
\end{large}

%%%%%%%%%%%%%%%%%%%%%%%%%%%%%%%%%%%%%%%%%%%%%%%%%%%%%%%%
\newpage 
\thispagestyle{empty}

\begin{abstract}
{\em
Este trabajo presenta el desarrollo de un sistema de predicción a corto plazo de series temporales meteorológicas, basado en técnicas de aprendizaje profundo, 
utilizando datos procedentes de múltiples estaciones meteorológicas de la isla de Tenerife. El objetivo principal es diseñar y
evaluar modelos capaces de generalizar a ubicaciones nuevas, no vistas durante el entrenamiento, sin necesidad de reentrenamiento. Este enfoque corresponde al
problema conocido como zero-shot, un ámbito aún poco explorado en la literatura de predicción de series temporales.

Se estudian distintas arquitecturas de redes neuronales, incluyendo modelos LSTM, CNN y un híbrido CNN-LSTM, comparándolos con enfoques tradicionales como ARIMA.
Se lleva a cabo un exhaustivo proceso de preprocesamiento de datos, que incluye la imputación de valores faltantes, codificación temporal y normalización,
así como la construcción de conjuntos de datos mediante ventanas deslizantes. Además, se realizan diversos experimentos para explorar técnicas y configuraciones
tanto en el tratamiento de los datos como en la arquitectura de los modelos, con el objetivo de optimizar los resultados.

Los experimentos muestran que los modelos LSTM alcanzan una alta precisión en horizontes de predicción cortos, superando a los modelos convencionales.
Finalmente, se implementa una aplicación web que permite a los usuarios generar pronósticos a partir de datos en tiempo real.
El sistema demuestra ser modular, escalable y con potencial para su aplicación práctica.

}

\begin{palabrasClave}
predicción de series temporales, aprendizaje profundo, LSTM, redes neuronales, meteorología
\end{palabrasClave}

\end{abstract}
%%%%%%%%%%%%%%%%%%%%%%%%%%%%%%%%%%%%%%%%%%%%%%%%%%%%%%%%%
\newpage 
\vspace*{200px}
\thispagestyle{empty}

\begin{summary}
{
This work presents the development of a short-term prediction system for meteorological time series, based on deep learning techniques, using data from multiple weather stations on the island of Tenerife. The main objective is to design and evaluate models capable of generalizing to new locations, not seen during training, without the need for retraining. This approach corresponds to the problem known as zero-shot, a domain still scarcely explored in the time series prediction literature.

Different neural network architectures are studied, including LSTM, CNN, and a CNN-LSTM hybrid model, comparing them with traditional approaches such as ARIMA. A thorough data preprocessing process is carried out, which includes missing value imputation, temporal encoding, and normalization, as well as the construction of datasets using sliding windows. Additionally, various experiments are conducted to explore techniques and configurations both in data treatment and model architecture, aiming to optimize the results.

The experiments show that LSTM models achieve high accuracy in short-term prediction horizons, outperforming conventional models. Finally, a web application is implemented that allows users to generate forecasts based on real-time data. The system proves to be modular, scalable, and with potential for practical application.
}

\em
\begin {keywords}
time series forecasting, deep learning, LSTM, neural networks, weather prediction
\end {keywords}

\end{summary}
%%%%%%%%%%%%%%%%%%%%%%%%%%%%%%%%%%%%%%%%%%%%%%%%%%%%%%%%%
\newpage{\pagestyle{empty}}
\thispagestyle{empty}

%%%%%%%%%%%%%%%%%%%%%%%%%%%%%%%%%%%%%%%%%%%%%%%%%%%%%%%%%
\pagestyle{myheadings} %my head defined by markboth or markright
% No funciona bien \markboth sin \char`\"twoside\char`\" en \documentclass, pero al
% ponerlo se dan un montón de errores de underfull \vbox, con lo que no se
% ha puesto.


%%Aqui debería poner el nombre del proyecto pero, como es muy grande no cabe y se ve feo en el PDF
\markboth{xxxxx}{}

%%%%%%%%%%%%%%%%%%%%%%%%%%%%%%%%%%%%%%%%%%%%%%%%%%%%%%%%%
%Numeracion en romanos
\renewcommand{\thepage}{\roman{page}}
\setcounter{page}{1}
\pagestyle{plain} 

%%%%%%%%%%%%%%%%%%%%%%%%%%%%%%%%%%%%%%%%%%%%%%%%%%%%%%%%%

\tableofcontents

%%%%%%%%%%%%%%%%%%%%%%%%%%%%%%%%%%%%%%%%%%%%%%%%%%%%%%%%%
\newpage{\pagestyle{empty}}

\listoffigures

%%%%%%%%%%%%%%%%%%%%%%%%%%%%%%%%%%%%%%%%%%%%%%%%%%%%%%%%%
\newpage{\pagestyle{empty}}

\listoftables

%%%%%%%%%%%%%%%%%%%%%%%%%%%%%%%%%%%%%%%%%%%%%%%%%%%%%%%%%%%%%%%%%%%%%%%%%%%%%%%
\newpage{\pagestyle{empty}}

%%%%%%%%%%%%%%%%%%%%%%%%%%%%%%%%%%%%%%%%%%%%%%%%%%%%%%%%%%%%%%%%%%%%%%%%%%%%%%%
\newpage
\thispagestyle{empty}

%Numeracion a partir del capitulo I
\renewcommand{\thepage}{\arabic{page}}
\setcounter{page}{1}
\pagestyle{plain}

\chapter{\LARGE Introducción}
\label{chapter:intro}

\section{Motivación del proyecto}
\subsection{Apartado Uno}
\begin{large}
Texto del apartado uno
As shown in \cite{smith2023}, the results are promising.

\begin{itemize}
   \item Item 1
   \item Item 2
   \item Item 3
   \item Item 4
\end{itemize}
\end{large}

\section{Planteamineto}
El TFG consistirá en el despliegue de una infraestructura tecnológica para la captura y el procesamiento de datos públicos, 
junto con el diseño, desarrollo y validación de un modelo predictivo que permita extraer información valiosa a partir de dichos datos.
 Este enfoque integrará tanto aspectos técnicos relacionados con la gestión y el análisis de datos como la aplicación de técnicas predictivas
  basadas en modelos de aprendizaje automático, con el fin de optimizar y automatizar la toma de decisiones en un contexto específico.
\begin{large}
\begin{itemize}
   \item Item 1
   \item Item 2
   \item Item 3
\end{itemize}
\end{large}

\section{Antecedentes y estado del arte}

\begin{large}
Bla, bla, bla.
\end{large}

\section{Objetivos}

\begin{large}
Bla, bla, bla.
\end{large}

\newpage

\begin{figure}[htb]
   \centering
   \includegraphics[width=0.8\linewidth]{images/figura_1}
   \caption{Ejemplo de figura}
   \label{chapter:intro}
\end{figure}

%%%%%%%%%%%%%%%%%%%%%%%%%%%%%%%%%%%%%%%%%%%%%%%%%%%%%%%%%%%%%%%%%%%%%%%%%%%%%%%
\newpage{\pagestyle{empty}}
\thispagestyle{empty}

\chapter{\LARGE Adquisición y preprocesado de datos}
\label{chapter:dos}


Se desea trabajar con series temporales sobre mediciones climatológicas. 
En concreto, se eligen las variables de temperatura del aire, humedad relativa y presión atmosférica en la superficie.
Dichas variables son estudiadas con frecuencia horaria, en el intervalo comprendido entre el 1 de marzo de 2023 y el 28 de febrero de 2025.

Es relevante el uso de datos en un período múltiplo del año, para asegurar que se capturan las variaciones estacionales y que el conjunto de datos está suficientemente equilibrado.

\section{Fuentes de los datos}

Se han empleado 2 fuentes para recopilar las mediciones: 
\begin{itemize}
    \item \textbf{Grafcan}: Cartográfica de Canarias, S.A. es una empresa pública de la Comunidad Autónoma de Canarias. Dispone de una red de estaciones meteorológicas cuyas
    mediciones son accesibles mediante una API REST de acceso gratuito previa solicitud de una clave\cite{grafcan_sensores}. 
    \item \textbf{Open-Meteo}: API pública de código abierto que proporciona datos de múltiples proveedores de meteorología. Este servicio no dispone de estaciones de medición
    propias, sino que recopila pronósticos de diferentes modelos de predicción climatológica. 

    Se emplea la API de predicciones pasadas\cite{open_meteo_api}. Se seleccionan los modelos ICON Global del servicio meteorológico alemán (DWD) y el modelo ARPEGE Europe de Météo-France. Ambos modelos se actualizan cada 3 horas. 
    Se explora la posibilidad de emplear as predicciones del modelo HAROME de la AEMET, pero no están disponibles de forma pública
\end{itemize}

Se eligen 4 ubicaciones de la isla de Tenerife con distintas características climáticas para el conjunto de entrenamiento y evaluación:
\begin{itemize}
    \item \textbf{San Cristóbal de La Laguna 1 (A)}: La Cuesta, 35 metros de altitud.
    \item \textbf{San Cristóbal de La Laguna 2 (B)}: La Punta del Hidalgo, 54m.
    \item \textbf{La Orotava (C)}: Camino de Chasna, 812m.
    \item \textbf{Arona (D)}: Punta de Rasca, 25m.
\end{itemize}

Así mismo, se escogen 2 ubicaciones para el conjunto de test, nunca vistas en el ajuste del modelo:
\begin{itemize}
    \item \textbf{Garachico (E)}: La Montañeta, 922 m.
    \item \textbf{Santa Cruz de Tenerife (F)}: Polígono Costa Sur, 92m.
\end{itemize}

Las ubicaciones han sido elegidas al contar con estaciones de medición de Grafcan.
Sus posiciones se muestran en la Figura \ref{mapa_estaciones}, con la letra indicada en la lista.
Se señaladan en rojo las estaciones de entrenamiento y en naranja las de test.

\begin{figure}[htb]
   \centering
   \includegraphics[width=0.6\linewidth]{images/mapa_estaciones}
   \caption{Mapa de las estaciones climatológicas Grafcan empleadas.}
   \label{mapa_estaciones}
\end{figure}

Inicialmente se valoró emplear las estaciones correspondientes a Los Cristianos, Santiago del Teide o la Punta de Teno, pero fueron descartadas por dos motivos: 
se detectó que existían períodos prolongados con datos faltantes en las mediciones de Grafcan. Algunas de ellas también exhibían poca correlación entre las mediciones
del servicio Grafcan y las de Open-Meteo, lo que podría afectar la calidad de los datos.

\bigskip

\section{Proceso de adquisición y almacenamiento}

Para automatizar la adquisición de datos, se emplea la herramienta de orquestación node-red, que permite crear flujos de información mediante nodos que realizan tareas específicas o ejecutan códgio de JavaScript.
En dicha herramienta se desarrollan dos paneles, uno para cada fuente de datos. 
Así mismo, dentro de cada panel se desarrollan dos flujos, uno para la adquisición de datos en un intervalo dado, y otro para la adquisición de datos en tiempo real, en particular, 
se establece la recogida de datos cada 6 horas.

\subsection{Flujos de adquisición de Grafcan}
Debido al funcionamiento de la API de Grafcan, se debe realizar una llamada para obtener la serie temporal de cada variable meteorológica de cada estación.
Posteriormente, se unen las series de cada estación en una única serie, que se almacena en una base de datos PostgreSQL. Este flujo está reflejado en la figura \ref{grafcan_flows}.
\begin{figure}[htb]
   \centering
   \includegraphics[width=1\linewidth]{images/node-red_grafcan.png}
   \caption{Flujo de adquisición de datos de Grafcan en node-red.}
   \label{grafcan_flows}
\end{figure}

Las mediciones de Grafcan se recogen apróximamente cada 10 minutos, si bien la frecuencia no es consistente y en ocasiones es mayor. Así mismo, 
los instantes de medición son independientes entre las variables estudiadas. Para manejar esta variabilidad, 
en este nivel los datos se agregan cada 10 minutos, usando la media de las mediciones del intervalo.

\subsection{Flujos de adquisición de Open-Meteo}
Existe una rama para obtener los datos del modelo ICON y otra para el modelo ARPEGE. Se establecen las coordenadas de cada ubicación como las de la estación de Grafcan seleccionada 
y se realiza una llamada a la API por cada localización y cada modelo, como se observa en la figura \ref{open-meteo_flows}. 
Los resultados se almacenan en una base de datos PostgreSQL.
\begin{figure}[htb]
   \centering
   \includegraphics[width=1\linewidth]{images/node-red_open-meteo.png}
   \caption{Flujo de adquisición de datos de Open-Meteo en node-red.}
   \label{open-meteo_flows}
\end{figure}

\subsection{Almacenamiento}
Se estudian distintas alternativas para el almacenamiento de los datos.
Se opta por emplear TimescaleDB, una extensión del popular sistema PostgreSQL
de bases de datos relacionales, especialmente adaptada para el manejo de series temporales. 

Se establece un servidor TimescaleDB en un contenedor Docker. Se configura una tabla para cada estación y cada fuente: Grafcan, Open-Meteo ICON y Open-Meteo ARPEGE. 
Cada tabla emplea como índice y clave primaria la fecha y hora de la medición, así como su zona horaria. Las otras columnas se corresponden a la temperatura media del aire
 en grados Celsius, la humedad relativa en porcentaje y la presión atmosférica en superficie medida en hPa.

Es importante señalar que las mediciones de Grafcan y Open-Meteo codifican las horas en UTC, en vez de la hora local, puesto que UTC es independiente a los cambios de horario
y de esta forma se mantiene la consistencia de los datos.

\section{Preprocesado}

Se desarrolla un cuaderno de Jupyter para realizar el preprocesado de los datos. El proceso descrito en este apartado
se aplica de forma separada para cada estación.

En primer lugar, se obtienen las series temporales de las 3 fuentes, Grafcan y los dos modelos de Open-Meteo, para el período entre el 1 de marzo de 2023 y el 28 de febrero de 2025.
Se agregan los datos con frecuencia horaria mediante la media. 

\textbf{Nota:} En la estación de Garachico, usada para el test, el período empleado es del 1 de marzo de 2024 al 28 de febrero de 2025, puesto que el período de 2023 tiene 
un gran número de datos faltantes.

\subsection{Visualización}
Se visualizan los datos de cada variable para cada año. Podemos ver ejemplos en las figuras \ref{visualizacion_1}, \ref{visualizacion_2} y \ref{visualizacion_3}.

Se observa claramente que en la presión atmosférica las mediciones de todas las fuentes son muy similares. Sin embargo, 
en la temperatura del aire y la humedad relativa se aprecian diferencias entre las distintas fuentes.
\begin{figure}
    \centering
    \includegraphics[width=.8\linewidth]{images/visualizacion_1.png}
    \caption{Visualización de la presión atmosférica en Arona durante 2023.}
    \label{visualizacion_1}
\end{figure}
\begin{figure}
    \centering
    \includegraphics[width=.8\linewidth]{images/visualizacion_2.png}
    \caption{Visualización de la temperatura del aire en Arona durante 2024.}
    \label{visualizacion_2}
\end{figure}
\begin{figure}
    \centering
    \includegraphics[width=.8\linewidth]{images/visualizacion_3.png}
    \caption{Visualización de la humedad relativa en Arona durante 2025.}
    \label{visualizacion_3}
\end{figure}

Cabe destacar que en la variable de humedad relativa se observa una gran variabilidad entre los datos de las distintas fuentes, que se constatará más adelante.

\subsection{Manejo de datos faltantes}
Se detectan los datos faltantes para cada fuente. Las estadísticas se muestran en la tabla \ref{tabla_datos_faltantes}. 
Es reseñable que el modelo ICON no muestra datos faltantes, mientras que el modelo ARPEGE tiene 35 horas faltantes consecutivas, en el período entre el 31 de diciembre de 2023 y el 1 de enero de 2024.

\begin{table}[htb]
    \centering
    \begin{tabular}{|c|c|c|c|}
        \hline
        Estación & Grafcan & Open-Meteo ICON & Open-Meteo ARPEGE \\
        \hline
        La Laguna 1 (La Cuesta) & 47 & 0 & 35 \\
        La Laguna 2(La Punta del Hidalgo) & 30 & 0 & 35 \\
        La Orotava & 3 & 0 & 35 \\
        Arona & 17 & 0 & 35 \\
        Garachico & 5 & 0 & 0 \\
        Santa Cruz de Tenerife & 0 & 0 & 46 \\
        \hline
    \end{tabular}
    \caption{Datos faltantes por estación y fuente de datos (en horas)}
    \label{tabla_datos_faltantes} 
\end{table}

Se decide imputar los datos faltantes mediante un método híbrido: si una secuencia de datos faltantes es menor a 5 horas se emplea 
un método de interpolación cúbica por tramos denominado PCHIP\cite{fritsch1980}, que preserva la forma de los datos y evita oscilaciones indeseadas.
Si la secuencia de datos faltantes es mayor a 5 horas, se copian los datos del día anterior en las mismas horas. 
Este tipo de imputación es común en el ámbito de las series temporales \cite{tawakuli2024}.

Los datos sintéticos son etiquetados como tales para poder ser identificados posteriormente.


\subsection{Selección de modelo de Open-Meteo}
Para cada variable se selecciona el modelo de Open-Meteo que mejor se ajusta a los datos de Grafcan.
Se emplean diversas métricas: los coeficiente de correlación de Pearson, Spearman y Kendall, así como el error cuadrático medio (MSE) y 
la distancia euclídea. Se elige el modelo que mejor resultados da en la mayoría de indicadores. Los resultados se muestran en la tabla \ref{tabla_modelos_seleccionados}.

\begin{table}[htb]
    \centering
    \begin{tabular}{|c|c|c|c|}
        \hline
        Estación & Temperatura del aire & Presión atmosférica & Humedad relativa \\
        \hline
        La Laguna 1 & ICON & ARPEGE & ICON \\
        La Laguna 2 & ARPEGE & ARPEGE & ARPEGE \\
        La Orotava & ICON & ARPEGE & ICON \\
        Arona & ICON & ARPEGE & ICON \\
        Garachico & ICON & ICON & ICON \\
        Santa Cruz de Tenerife & ICON & ARPEGE & ICON \\
        \hline
    \end{tabular}
    \caption{Modelo de Open-Meteo seleccionado para cada variable y estación}
    \label{tabla_modelos_seleccionados}
\end{table}
Observamos que, en general, el modelo ICON es el que mejor se ajusta a las variables de temperatura del aire y humedad relativa,
 mientras que el modelo ARPEGE es el que mejor se ajusta a la presión atmosférica.

 Resulta reseñable señalar que la diferencia entre los modelos de Open-Meteo y Grafcan es mucho mayor en la variable 
 de humedad relativa que en las otras. Esto puede ser relevante más adelante, de cara al rendimiento de los modelos de predicción. 

 \bigskip
 Tras selecciona el modelo de Open-Meteo para cada variable, se crea un dataset unificado con las 3 variables. De esta forma se dispone 
 de un dataset de Open-Meteo y otro de Grafcan con las 3 variables para cada estación.

\subsection{Detección de valores anómalos}
Para la detección de valores anómalos, en primer lugar se empea el método del rango intercuartílico (IQR). 
Sin embargo, se observa en los histogramas que la distribución de los datos no es puramente gaussiana, existiendo sesgos y colas largas.
Por ejemplo, en la figura \ref{histogram_1} se observa que la humedad presenta un sesgo a la izquierda, mientras que en la figura 
\ref{histogram_2} se observa que la temperatura presenta una cola larga a la derecha.

\begin{figure}
    \centering
    \includegraphics[width=.5\linewidth]{images/histogram_humidity.png}
    \caption{Histograma de la humedad relativa en Arona.}
    \label{histogram_1}
\end{figure}

\begin{figure}
    \centering
    \includegraphics[width=.5\linewidth]{images/histogram_temperature.png}
    \caption{Histograma de la temperatura del aire en Arona.}
    \label{histogram_2}
\end{figure}

Por esto, se decide emplear como método de detección de anomalías el de los K vecinos más cercanos (KNN), 
una alternativa robusta que permite detectar anomalías en distribuciones no gaussianas \cite{gu2019}.

Se aplica el método de KNN a cada variable por separado, con k=10 y límite de distancia 4 veces la desviación típica. 
Los valores anómalos son etiquetados como tales para poder ser identificados posteriormente. Se pueden observar ejemplos 
de las distancias en la figura \ref{knn_distances}. En las figuras \ref{histogram_knn_humidity} y \ref{histogram_knn_temperature} se muestran ejemplos de los valores anómalos detectados. 

Nota: Se muestran los valores anómalos indepentientemente de la variable respecto a la que se detectaron.

\begin{figure}
    \centering
    \includegraphics[width=.5\linewidth]{images/knn_distances_temperature_arona.png}
    \caption{Distancias media de los K vecinos más cercanos para la temperatura del aire en Arona.}
    \label{knn_distances}
\end{figure}

\begin{figure}
    \centering
    \includegraphics[width=.5\linewidth]{images/histogram_humidity_knn.png}
    \caption{Histograma de la humedad relativa en Arona con outliers detectados con knn.}
    \label{histogram_knn_humidity}
\end{figure}

\begin{figure}
    \centering
    \includegraphics[width=.5\linewidth]{images/histogram_temperature_knn.png}
    \caption{Histograma de la temperatura del aire en Arona con outliers detectados con knn.}
    \label{histogram_knn_temperature}
\end{figure}

\subsection{Exploración de frecuencias - Dominio de Fourier}
Se realiza un análisis de Fourier para cada variable. De esta forma, se pueden observar las frecuencias dominantes en los datos. Este análisis
es relevante para detectar las frecuencias a emplear en la codificación de la información temporal, que se aborda en el siguiente apartado.

Se emplea la transformada rápida de Fourier (FFT), se filtran las frecuencias positivas, se acota a las frecuencias mayores a \(10^{-3}\) 
 (unos 16,66... minutos) y se grafican haciendo uso de escala logarítmica en el eje X. Se pueden observar ejemplos en las figuras \ref{fft_temperature},
 y \ref{fft_pressure}. Se señalan las 5 frecuencias de mayor magnitud con un punto en rojo. 

\begin{figure}
    \centering
    \includegraphics[width=.5\linewidth]{images/fft_temperature.png}
    \caption{Transformada rápida de Fourier de la temperatura del aire en Arona.}
    \label{fft_temperature}
\end{figure}

 \begin{figure}
    \centering
    \includegraphics[width=.5\linewidth]{images/fft_pressure.png}
    \caption{Transformada rápida de Fourier de la presión atmosférica en Arona.}
    \label{fft_pressure}
\end{figure}



En la temperatura y humedad destacan las frecuencias de 24 y 8772 horas, correspondiente esta última a 365,5 días, lo que es razonable considerando que de los 
2 años de datos, uno es bisiesto. Respecto a la presión atmosférica, la más relevante es la de 12 horas, algo debido a la naturaleza de esta variable, 
que presenta este ciclo debido a un fenómeno conocido como mareas térmicas \cite{ChapmanLindzen1970}. Así mismo, la presión también presenta un pico en la frecuencia anual y la de medio año.


\subsection{Codificación de la información temporal}
Se decide codificar la información temporal mediante el uso de senos y cosenos de las frecuencias dominantes.
En base a los resultados del análisis de Fourier, se emplean la frecuencia de 24 horas y del año, teniendo especial cuidado para detectar
    si el año es bisiesto o no. De esta forma, se añaden 4 variables adicionales al dataset: sin(dia), cos(diá), sin(año) y cos(año).

Se estudia la frecuencia semanal, pero se descarta al existir poca correlación.

\subsection{Estudio de correlación}
Se estudia la correlación entre las distintas variables que conforman el dataset mediante una matriz de correlación con el coeficiente de Pearson.
Se observa que las variables climáticas tienen una correlación de entorno a 0.3, lo que indica una correlación baja \ref{correlation_map}.

\begin{figure}
    \centering
    \includegraphics[width=.5\linewidth]{images/correlation_heatmap.png}
    \caption{Mapa de correlación entre las variables del dataset.}
    \label{correlation_map}
\end{figure}

\section{Creación de ventanas de datos}





%%%%%%%%%%%%%%%%%%%%%%%%%%%%%%%%%%%%%%%%%%%%%%%%%%%%%%%%%%%%%%%%%%%%%%%%%%%%%%%
\newpage{\pagestyle{empty}}
\thispagestyle{empty}

\chapter{\LARGE Modelos de predicción}
\label{chapter:tres}

Los capítulos intermedios servirán para cubrir los siguientes aspectos: antecedentes, problemática o estado del arte, objetivos, fases y desarrollo del proyecto.

\bigskip
Bla, Bla, Bla, .....

\section{Comparativa inicial}
\section{Estudio del tamaño de las ventanas}
\section{Uso de ruido}

%%%%%%%%%%%%%%%%%%%%%%%%%%%%%%%%%%%%%%%%%%%%%%%%%%%%%%%%%
\newpage{\pagestyle{empty}}
\thispagestyle{empty}

\chapter{\LARGE Despliegue}
\label{chapter:cuatro}


Se desea realizar un despliegue del proyecto, de forma que los modelos de predicción sean accesibles mediante una interfaz web. Específicamente, se plantea que cualquier usuario 
pueda seleccionar una de las estaciones de GRAFCAN, y a partir de la misma, que la aplicación web devuelva la predicción de alguna de las variables meteorológicas en las próximas horas. 
Se decide arbitrariamente acotar la predicción a la variable de temperatura y a las próximas 6 horas.

Se realiza una interfaz web sencilla como front-end, y un back-end que maneje la carga de trabajo de predicción. Las predicciones se ejecutan de forma asíncrona 
para no bloquear la experiencia del usuario y permitir un cierto nivel de paralelismo.

\section{Front-end}
Tras valorar distintas opciones, se decide emplear React como framework para el front-end. Se elige por su facilidad de uso y la gran cantidad de librerías que existen para este.
Se emplea Typescript para la codificación, al ser una alternativa más robusta a Javascript, y que permite detectar errores de forma más sencilla.

Se crea una aplicación de una sola página (Single Page Application) que permite al usuario buscar y seleccionar una estación de GRAFCAN. Una vez seleccionada,
se envía una petición al back-end para obtener la predicción.

Las características de la aplicación son las siguientes:
\begin{itemize}
    \item \textbf{Selector de estaciones}: Permite al usuario seleccionar una estación de GRAFCAN. Se agrupan las estaciones por ubicación para facilitar la búsqueda.
    \item \textbf{Visualización de sensores}: Una vez seleccionada la estación, se muestran los sensores disponibles en la misma. 
    \item \textbf{Indicadores visuales de estado de la predicción}: Se utilizan indicadores visuales para mostrar el estado de la predicción. Estos indicadores permiten al usuario saber si la predicción está en curso, ha fallado o se ha completado con éxito.
\end{itemize}


Para el desarrollo de la aplicación es necesario crear una serie de elementos denominados \char`\"componentes". Estos componentes son bloques de código que 
permiten crear una interfaz de usuario modular y reutilizable. En este caso, se crean, entre otros:
\begin{itemize}
    \item \textbf{Dashboard}: Componente principal de la aplicación. Se encarga de gestionar el estado de la aplicación y de renderizar los demás componentes.
    \item \textbf{Selector de estaciones}: permite al usuario seleccionar una estación de GRAFCAN, ordenadas por ubicación geográfica.
    \item \textbf{Gráficas de sensor}: Se utilizan para mostrar los datos históricos de la estación seleccionada. 
    \item \textbf{Estado de predicción}: Indicadores permiten al usuario saber si la predicción está en curso, ha fallado o se ha completado con éxito.
    \item \textbf{Información de estación}: Muestra información detallada de la estación seleccionada, incluyendo los sensores disponibles y su estado.
\end{itemize}

La figura \ref{frontend_loading} muestra un ejemplo de la interfaz web mientras se recuperan los datos de una estación, la figura 
\ref{frontend_loaded} contiene el resultado y en \ref{frontend_prediction} se muestra el resultado de una predicción.

\begin{figure}[H]
    \centering
    \includegraphics[width=0.9\textwidth]{images/frontend_loading.png}
    \caption{Interfaz web. Carga de datos de una estación}
    \label{frontend_loading}
\end{figure}

\begin{figure}[H]
    \centering
    \includegraphics[width=0.9\textwidth]{images/frontend_loaded.png}
    \caption{Interfaz web. Gráfica con datos de una estación}
    \label{frontend_loaded}
\end{figure}

\begin{figure}[H]
    \centering
    \includegraphics[width=0.9\textwidth]{images/frontend_prediction_6.png}
    \caption{Interfaz web. Resultado de una predicción}
    \label{frontend_prediction}
\end{figure}


\section{Back-end}

\begin{itemize}
    \item \textbf{FastAPI}: Se trata de un framework web para construir APIs en Python, que facilita crear servicios RESTful con alto rendimiento. Se emplea para gestionar las peticiones de predicción.
    \item \textbf{Worker Celery}: Como nodo trabajador se emplea Celery, una biblioteca de Python para ejecutar tareas asíncronas y distribuir trabajos en segundo plano, ideal para procesar tareas largas o programadas sin bloquear la aplicación principal. 
    Se utiliza como sistema de tareas asíncronas para evitar bloquear el servidor web durante la ejecución de las predicciones. El worker carga el modelo LSTM previamente entrenado al iniciarse, utilizando los pesos almacenados localmente. 
    \item \textbf{Redis}: Se utiliza como broker de tareas para Celery. Permite almacenar y gestionar las tareas encoladas, facilitando la comunicación entre el servidor web y los workers.
\end{itemize}

Esta arquitectura, que se muestra en la figura \ref{deploy_scheme}, facilita el despliegue, escalabilidad y mantenimiento del sistema, permitiendo además la ejecución paralela de múltiples workers para atender una mayor carga de solicitudes.
\begin{figure}[H]
    \centering
    \includegraphics[width=0.9\textwidth]{images/esquema_despliegue.png}
    \caption{Esquema de la lógica del despliegue}
    \label{deploy_scheme}
\end{figure}

La API es responsable de validar y transformar los datos de entrada, preparar la estructura requerida por el modelo de predicción y coordinar la ejecución de las tareas de inferencia mediante Celery.

Los endpoints disponibles en la API son:

\begin{itemize}
    \item \textbf{POST $/$predict}:  recibe datos estructurados correspondientes a observaciones recientes de una estación. Estos datos se procesan para generar un tensor con forma [1, T, F], donde T representa el número de intervalos temporales considerados y F = 7 corresponde a las características de entrada (cuatro variables temporales codificadas mediante funciones seno y coseno para capturar patrones cíclicos diarios y anuales, y tres variables provenientes de sensores). A continuación, la solicitud es encolada como una tarea en Celery para su procesamiento.
    \item \textbf{GET $/$predict$/${job\_id}}: permite consultar el estado de la tarea asociada al identificador job\_id. En caso de que la predicción ya esté disponible, devuelve el resultado generado.
\end{itemize}
El procesamiento interno del back-end incluye la extracción de características temporales relevantes y la conversión de los datos a tensores NumPy que alimentan el modelo de predicción.


El flujo general de una petición de predicción es el siguiente:

\begin{itemize}
    \item El usuario selecciona una estación de GRAFCAN y un sensor específico en la interfaz web.
    \item El front-end recopila las observaciones de los últimos 48 intervalos horarios y envía esta información al back-end mediante el endpoint POST /predict.
    \item El back-end valida y transforma los datos, encolando la tarea en Celery.
    \item El worker procesa la tarea, ejecutando el modelo LSTM para obtener la predicción.
    \item El front-end consulta periódicamente el estado y resultado de la tarea mediante el endpoint GET /predict/{job\_id}.
    \item  Finalmente, el resultado se presenta visualmente, superponiéndose a las gráficas de las observaciones originales del sensor.
\end{itemize}



%%%%%%%%%%%%%%%%%%%%%%%%%%%%%%%%%%%%%%%%%%%%%%%%%%%%%%%%%
\newpage{\pagestyle{empty}}
\thispagestyle{empty}

\chapter{\LARGE Conclusiones y líneas futuras}
\label{chapter:Resultados}


En este trabajo se ha desarrollado un sistema de predicción de variables meteorológicas, capaz de generalizar y realizar predicciones de ubicaciones no vistas. 
Se han estudiado diversas alternativas en el ámbido del aprendizaje profundo, y se ha observado que los modelos LSTM son los que mejores resultados brindan de entre los modelos estudiados. Además,
se ha comprobado que estos resultados son buenos. Son mejores que los de modelos tradicionales para predicción de series temporales, y superan a los resultados de otras investigaciones 
recientes.

Se ha elaborado todo un flujo de trabajo, desde la obtención de datos, su tratamiento y limpieza, hasta la creación de los modelos de predicción. Y se han 
realizado extensas pruebas de distintas teorías y configuraciones, con el fin de obtener los mejores resultados para este problema.

Una de las consideraciones más llamativas es que, estudiando el tamaño de la ventana de datos pasados, se observa que empleando pocas mediciones, máximo un día, se obtienen mejores predicciones que
con períodos más extensos. Esto es particularmente útil puesto que permite reducir el número de mediciones necesarias para realizar una predicción, haciendo los modelos 
más prácticos para su uso en la vida real.

Por otra parte, se ha desarrollado una aplicación web que pone al alcance de cualquier usuario realizar predicciones de forma sencilla. 
Esta aplicación permite al usuario seleccionar una estación de Grafcan y obtener la predicción de temperatura para las próximas horas.
Esto supone una prueba de concepto de la aplicabilidad del sistema desarrollado, y su potencial para ser utilizado en la vida real. La solución diseñada
es modular y escalable, siendo una propuesta seria para su uso en producción real.

\section{Líneas futuras}

Se ha constatado que el uso de un mayor número de puntos de medición en la construcción del conjunto de entrenamiento mejora los resultados de predicción.
La escalada inteligente de este conjunto, incluyendo información geográfica de las ubiaciones como la altitud y coordenadas, así como la construcción de una malla 
de puntos de medición puede suponer una línea de trabajo futura.

Por otra parte, se ha observado que la utilización de covairables adicionales, como la humedad o la presión atmosférica, en el caso de la temperatura, mejoran los resultados de predicción.
El estudio de estas covariables y la inclusión de otras puede suponer una mejora en los resultados de predicción.

Finalmente, si bien se han estudiado diversos modelos de aprendizaje profundo, existen múltiples alternativas que no se han estudiado en este trabajo.
El estudio de modelos de última generación, como los transformers, supone otro ámbito de trabajo futuro.

%%%%%%%%%%%%%%%%%%%%%%%%%%%%%%%%%%%%%%%%%%%%%%%%%%%%%%%%%
\newpage{\pagestyle{empty}}
\thispagestyle{empty}

\chapter{\LARGE Summary and Conclusions}
\label{chapter:Conclusiones}

In this work, a weather variable forecasting system has been developed, capable of generalizing and making predictions for previously unseen locations. Various alternatives in the field of deep learning have been studied, and it has been observed that LSTM models provide the best results among the models evaluated. Furthermore, these results have proven to be strong—surpassing those of traditional time series forecasting models, and even outperforming recent research efforts.

A complete workflow has been established, from data acquisition, processing, and cleaning, to the design and implementation of forecasting models. Extensive testing of different hypotheses and configurations has been conducted, with the aim of achieving optimal performance for the problem at hand.

One of the most notable findings is that, when analyzing the size of the input data window, better predictions are obtained using a small number of past measurements—up to one day—than with longer periods. This is particularly useful, as it allows the number of required measurements for making a prediction to be reduced, making the models more practical for real-world applications.

In addition, a web application has been developed, enabling any user to easily generate forecasts. This application allows the user to select a Grafcan station and obtain a temperature forecast for the upcoming hours. This serves as a proof of concept for the applicability of the developed system and its potential for real-world deployment. The solution is modular and scalable, representing a serious proposal for production use.

\section{Future Work}

It has been demonstrated that increasing the number of measurement points used in the construction of the training set improves forecasting accuracy. The intelligent scaling of this set, including geographic information such as altitude and coordinates, as well as the construction of a mesh of measurement points, represents a promising direction for future research.

Additionally, it has been observed that the use of additional covariates—such as humidity or atmospheric pressure, in the case of temperature forecasting—improves prediction accuracy. Further exploration of these and other covariates may enhance forecasting results.

Finally, although several deep learning models have been evaluated, many other approaches were not explored in this study. The investigation of state-of-the-art models, such as Transformers, constitutes another promising area for future work.

%%%%%%%%%%%%%%%%%%%%%%%%%%%%%%%%%%%%%%%%%%%%%%%%%%%%%%%%%
\newpage{\pagestyle{empty}}
\thispagestyle{empty}

\chapter{\LARGE Presupuesto}
\label{chapter:presupuesto}

Este capítulo es obligatorio. Toda memoria de Trabajo de Fin de Grado debe incluir un presupuesto.

\section{Gastos de desarrollo}
Los gastos estimados en los que se ha incurrido para el desarrollo del proyecto, con una duración de de cuatro meses,
 se resumen en la tabla \ref{tabla_presupuesto_desarrollo}. 

\begin{table}[h]
   \centering
   \small
   \caption{Resumen de gastos}
   \begin{tabular}{|l|l|c|c|c|}
      \hline
      \textbf{Concepto} & \textbf{Descripción} & \textbf{Unidades} & \textbf{Coste unitario} & \textbf{Coste total} \\ \hline
      Recursos humanos & Ingeniero Informático & 200 h & 20 €/h & 4000 € \\ \hline
      Ordenador & Para el desarrollo del proyecto & 1 & 1000 € & 1000 € \\ \hline
      Servidor con GPU & Para el entrenamiento de modelos & 1 & 600 € & 600 € \\ \hline
      \textbf{Total} & & & & \textbf{5600 €} \\ \hline
   \end{tabular}
   \label{tabla_presupuesto_desarrollo}
\end{table}

\section{Gastos de despliegue}
Si se decidiera desplegar el sistema para su uso en producción, se estiman los gastos indicados
en la tabla \ref{tabla_presupuesto_despliegue}.

\begin{table}[h]
   \centering
   \small
   \caption{Resumen de gastos}
   \begin{tabular}{|l|l|c|c|c|}
      \hline
      \textbf{Concepto} & \textbf{Descripción} & \textbf{Coste} \\ \hline
      Servidor con GPU & Para la ejecución de modelos  & 600 € \\ \hline
      Servidor web & Para el despliegue del sistema  & 200 € \\ \hline
      Dominio y hosting & Para el acceso al sistema  & 50 € \\ \hline
      \textbf{Total}  & & \textbf{850 €} \\ \hline
   \end{tabular}
   \label{tabla_presupuesto_despliegue}
\end{table}

%%%%%%%%%%%%%%%%%%%%%%%%%%%%%%%%%%%%%%%%%%%%%%%%%%%%%%%%%
% \newpage{\pagestyle{empty}\cleardoublepage}
% \thispagestyle{empty}

% \begin{appendix}

% \chapter{\LARGE Título del Apéndice 1}
% \label{appendix:1}
% \input{apendice1.tex}

% \chapter{\LARGE Título del Apéndice 2}
% \label{appendix:2}
% \input{apendice2.tex}

% \end{appendix}

%%%%%%%%%%%%%%%%%%%%%%%%%%%%%%%%%%%%%%%%%%%%%%%%%%%%%%%%%%
\bibliographystyle{IEEEtran}
\begin{thebibliography}{X}
% Aquí figurará la bibliografía
\bibitem{mcculloch1943} 
W. S. McCulloch and W. Pitts, \char`\"A logical calculus of the ideas immanent in nervous activity,\char`\" Bull. Math. Biophys., vol. 5, no. 4, pp. 115–133, 1943. https://doi.org/10.1007/BF02478259

\bibitem{hebb1949}
D. O. Hebb, \textit{The organization of behavior: A neuropsychological theory}. New York, NY, USA: Wiley, 1949.

\bibitem{rosenblatt1958} 
F. Rosenblatt, \char`\"The perceptron: A probabilistic model for information storage and organization in the brain,\char`\" Psychol. Rev., vol. 65, no. 6, pp. 386–408, 1958. https://doi.org/10.1037/h0042519

\bibitem{werbos1982} 
P. J. Werbos, \char`\"Applications of advances in nonlinear sensitivity analysis,\char`\" in \textit{System Modeling and Optimization}, R. F. Drenick and F. Kozin, eds., vol. 38. Berlin, Germany: Springer-Verlag, 1982, pp. 762–770. https://doi.org/10.1007/BFb0006203

\bibitem{elman1990}
J. L. Elman, \char`\"Finding structure in time,\char`\" Cogn. Sci., vol. 14, no. 2, pp. 179–211, 1990. https://doi.org/10.1207/s15516709cog1402\_1

\bibitem{hochreiter1997}
S. Hochreiter and J. Schmidhuber, \char`\"Long short-term memory,\char`\" Neural Comput., vol. 9, no. 8, pp. 1735–1780, 1997. https://doi.org/10.1162/neco.1997.9.8.1735

\bibitem{cho2014}
K. Cho, B. Van Merriënboer, C. Gulcehre, D. Bahdanau, F. Bougares, H. Schwenk, and Y. Bengio, \char`\"Learning phrase representations using RNN encoder–decoder for statistical machine translation,\char`\" in \textit{Proc. 2014 Conf. Empirical Methods in Natural Language Processing}, 2014, pp. 1724–1734. https://doi.org/10.3115/v1/D14-1179

\bibitem{vaswani2017}
A. Vaswani, N. Shazeer, N. Parmar, J. Uszkoreit, L. Jones, A. N. Gómez, Ł. Kaiser, and I. Polosukhin, \char`\"Attention is all you need,\char`\" in \textit{Advances in Neural Information Processing Systems}, vol. 30, 2017, pp. 5998–6008. https://doi.org/10.48550/arXiv.1706.03762

\bibitem{shi2015}
X. Shi, Z. Chen, H. Wang, D.-Y. Yeung, W.-K. Wong, and W.-c. Woo, \char`\"Convolutional LSTM network: A machine learning approach for precipitation nowcasting,\char`\" in \textit{Advances in Neural Information Processing Systems}, vol. 28, 2015, pp. 802–810. https://doi.org/10.48550/arXiv.1506.04214

\bibitem{weyn2019}
J. A. Weyn, D. R. Durran, and R. Caruana, \char`\"Improving data-driven global weather prediction using deep convolutional neural networks on a cubed sphere,\char`\" J. Adv. Model. Earth Syst., vol. 12, no. 9, 2020. https://doi.org/10.1029/2020MS002109

\bibitem{fu2019}
Y. Fu, F. Wang, Z. Shao, C. Yu, Y. Li, Z. Chen, Z. An, and Y. Xu, \char`\"LightWeather: Harnessing absolute positional encoding to efficient and scalable global weather forecasting,\char`\" arXiv preprint arXiv:2408.09695, 2024. https://doi.org/10.48550/arXiv.2408.09695

\bibitem{deznabi2024}
I. Deznabi, P. Kumar, and M. Fiterau, \char`\"Zero-shot microclimate prediction with deep learning,\char`\" arXiv preprint arXiv:2401.02665, 2024. https://doi.org/10.48550/arXiv.2401.02665

\bibitem{grafcan_sensores} 
Cartográfica de Canarias, S.A., Sistema de Observación Meteorológica de Canarias [Online]. Available: https://sensores.grafcan.es/ [Accessed: 12-May-2025]

\bibitem{open_meteo_api}
Open-Meteo, Historical Forecast API [Online]. Available: https://open-meteo.com/en/docs/historical-forecast-api [Accessed: 12-May-2025]

\bibitem{fritsch1980}
F. N. Fritsch and R. E. Carlson, \char`\"Monotone piecewise cubic interpolation,\char`\" SIAM J. Numer. Anal., vol. 17, no. 2, pp. 238–246, 1980. https://doi.org/10.1137/0717021

\bibitem{tawakuli2024}
A. Tawakuli, B. Havers-Zulka, V. Gulisano, and D. Kaiser, \char`\"Survey: Time-series data preprocessing: A survey and an empirical analysis,\char`\" J. Eng. Res., 2024, advance online publication. https://doi.org/10.1016/j.jer.2024.02.018

\bibitem{gu2019}
X. Gu, L. Akoglu, and A. Rinaldo, \char`\"Statistical analysis of nearest neighbor methods for anomaly detection,\char`\" in \textit{Advances in Neural Information Processing Systems}, vol. 32, Curran Associates, Inc., 2019, pp. 10921–10931. https://doi.org/10.48550/arXiv.1907.03813

\bibitem{ChapmanLindzen1970}
S. Chapman and R. S. Lindzen, \textit{Atmospheric tides}. Dordrecht, The Netherlands: D. Reidel Publishing Company, 1970. https://doi.org/10.1007/978-94-010-3399-2

\bibitem{graves_schmidhuber2005}
A. Graves and J. Schmidhuber, \char`\"Framewise phoneme classification with bidirectional LSTM networks,\char`\" Neural Networks, vol. 18, no. 5–6, pp. 602–610, 2005. https://doi.org/10.1016/j.neunet.2005.06.042

\bibitem{bahdanau2014}
D. Bahdanau, K. Cho, and Y. Bengio, \char`\"Neural machine translation by jointly learning to align and translate,\char`\" arXiv preprint arXiv:1409.0473, 2014. https://doi.org/10.48550/arXiv.1409.0473

\bibitem{haque2021}
E. Haque, S. Tabassum, and E. Hossain, \char`\"A comparative analysis of deep neural networks for hourly temperature forecasting,\char`\" IEEE Access, vol. 9, pp. 160646–160660, 2021. https://doi.org/10.1109/ACCESS.2021.3131533

\end{thebibliography}
%%%%%%%%%%%%%%%%%%%%%%%%%%%%%%%%%%%%%%%%%%%%%%%%%%%%%%%%%%

\end{document}

