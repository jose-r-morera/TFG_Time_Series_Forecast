Se desarrollan 4 modelos: un modelo ARIMA como base de comparación, un modelo LSTM, un modelo CNN y un híbrido LSTM-CNN.

\section{ARIMA}


\section{Modelos de aprendizaje profundo}
Existen diferentes frameworks para el desarrollo de modelos de aprendizaje profundo, destacando TensorFlow, PyTorch o JAX/Flax como los más extendidos.
Después de evaluar alternativas, se opta por TensorFlow debido a su amplia comunidad de usuarios y su extensa documentación, así comopor su integración con Keras, 
una biblioteca de alto nivel que facilita la creación de modelos de aprendizaje profundo.

\subsection{LSTM}

\subsection{CNN}

\subsection{LSTM-CNN}


\section{Comparativa inicial}

\section{Estudios empíricos}

\subsection{Número de estaciones}

\subsection{Tamaño de la ventana}

\subsection{Uso de ruido}

\section{Resultados}
\subsection{Temperatura del aire}
\subsection{Humedad relativa}
\subsection{Presión atmosférica}

