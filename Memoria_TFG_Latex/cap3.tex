Se desarrollan 4 modelos: un modelo ARIMA como base de comparación, un modelo LSTM, un modelo CNN y un híbrido LSTM-CNN.

\section{ARIMA}
Los modelos ARIMA (Autoregressive Integrated Moving Average) son una tipo de modelos estadísticos utilizados históricamente para el análisis y la predicción de series temporales
univariable. 

Utilizamos como base de comparación los resultados de un ARIMA para cada serie temporal: cada variable en cada ubicación y cada fuente de datos.

Para emplear un modelo ARIMA, es necesario ajustar cada uno de sus parámetros, que son:
\begin{itemize}
    \item $p$: el número de términos autorregresivos (AR).
    \item $d$: el número de diferencias necesarias para hacer la serie estacionaria (I).
    \item $q$: el número de términos de media móvil (MA).
\end{itemize}

\subsection{Calibración de térmios autorregresivos}
Para determinar el número de términos autorregresivos, se utiliza la función \texttt{plot\_pacf} de la librería \texttt{statsmodels}, 
que permite visualizar la función de autocorrelación parcial (PACF) de la serie temporal. Observamos un ejemplo de los resultados en la figura \ref{arima_pacf}. 

\begin{figure}[H]
\centering
\includegraphics[width=0.8\textwidth]{img/arima_pacf.png}
\caption{Ejemplo de la función de autocorrelación parcial (PACF) de una serie temporal}
\label{arima_pacf}
\end{figure}

Analíticamente, se emplea el algoritmo descrito en \ref{pacf_ar_order}. Se determina $p$ = 4.

\begin{figure}[H]
{\small
 \hrule \
 {\bf\small Pseudocódigo Selección de Orden AR vía PACF}
 \hrule
\begin{center}
\begin{tabbing}
\ 1: {\bf Fun}\={\bf ción} seleccionar\_orden\_AR($serie$, $max\_lags$): \\
\ 2: \> \# 1. Calcular la PACF hasta $max\_lags$ retardos \\
\ 3: \> $pacf\_valores$ = PACF($serie$, $nlags$=$max\_lags$) \\
\ 4: \> \# 2. Definir intervalo de confianza al 95\% \\
\ 5: \> $intervalo\_confianza$ = 1.96 / raiz\_cuadrada(longitud($serie$)) \\
\ 6: \> \# 3. Buscar el primer retardo no significativo \\
\ 7: \> {\bf Para} \= $lag$ {\bf en} 1 \dots $max\_lags$: \\
\ 8: \> \> {\bf Si} \= valor\_absoluto($pacf\_valores[lag]$) $<$ $intervalo\_confianza$: \\
\ 9: \> \> \> imprimir("El mejor orden AR sugerido por la PACF es:", $lag-1$) \\
\ 10: \> \> \> {\bf Retornar} $lag-1$ \\
\ 11: \> \# 4. Si todos los retardos son significativos \\
\ 12: \> imprimir("Todos los retardos hasta", $max\_lags$, "son significativos.) \\
\ 13: \> {\bf Retornar} $max\_lags$ \\
\end{tabbing}
\end{center}
\hrule
}
\caption{Pseudocódigo para Selección de Orden AR usando PACF}
\label{pacf_ar_order}
\end{figure}

\subsection{Calibración de }

\section{Modelos de aprendizaje profundo}
Existen diferentes frameworks para el desarrollo de modelos de aprendizaje profundo, destacando TensorFlow, PyTorch o JAX/Flax como los más extendidos.
Después de evaluar alternativas, se opta por TensorFlow debido a su amplia comunidad de usuarios y su extensa documentación, así comopor su integración con Keras, 
una biblioteca de alto nivel que facilita la creación de modelos de aprendizaje profundo.

\subsection{LSTM}

\subsection{CNN}

\subsection{LSTM-CNN}


\section{Comparativa inicial}
Realizar una comparativa justa entre los modelos ARIMA y los modelos de aprendizaje profundo es imposible, debido a que tienen naturalezas distintas.
Los 

\section{Estudios empíricos}

\subsection{Número de estaciones}

\subsection{Tamaño de la ventana}

\subsection{Uso de ruido}

\section{Resultados}
\subsection{Temperatura del aire}
\subsection{Humedad relativa}
\subsection{Presión atmosférica}

