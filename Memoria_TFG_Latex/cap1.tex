\section{Motivación del proyecto}
La predicción de series temporales es un campo de estudio que ha cobrado gran relevancia en los últimos años, especialmente en el ámbito del aprendizaje automático y el aprendizaje profundo. 
La capacidad de anticipar eventos futuros a partir de datos históricos es fundamental en diversas áreas, como la economía, la meteorología, la salud y la ingeniería. 

El campo de la meteorología, en particular, ha aumentado su relevancia en los útlimos años debido al aumento de fenómenos climáticos extremos y su impacto en la sociedad. 
Prever con antelación la evolución de variables meteorológicas no solo permite planificar recursos, sino también contribuir a la prevención de desastres naturales.

En este contexto, el uso de redes neuronales ha demostrado ser una herramienta poderosa para abordar problemas complejos de predicción.
Gracias a los avances en capacidad de cómputo y la disponibilidad de grandes volúmenes de datos, arquitecturas como LSTM, GRU y Transformers se han consolidado como soluciones de alto rendimiento. 
Estas redes no solo capturan patrones temporales de manera eficiente, sino que también pueden adaptarse dinámicamente a cambios en el comportamiento de la serie, mejorando la precisión y la robustez de las predicciones.


\section{Planteamineto}
En este trabajo se pretende emplear mediciones de múltiples estaciones meteorológicas de la isla de Tenerife con el fin de desarrollar un modelo de predicción climatológica a corto plazo.
Se busca que dicho modelo sea capaz de generalizar más allá de las estaciones de entrenamiento. Esto es, que a partir de mediciones meteorológicas de cualquier origen, el modelo sea capaz de 
generar predicciones a corto plazo (3, 6 o 12 horas) de gran calidad.
Se deciden considerar 3 variables meteorológicas: temperatura, humedad y presión atmosférica.
La elección de estas variables se basa en su relevancia para la predicción del clima y la existencia de registros en las estaciones meteorológicas de Tenerife.

Se establece un especial énfasis en el tratamiento de los datos, estudiándose exhaustivamente diversas técnicas.

La idea principal pasa por la creación de ventanas de información, que son secuencias de datos consecutivos agrupados 

Como parte del trabajo se contempla el despliegue de una infraestructura tecnológica que permita la captura y el procesamiento de datos públicos,
 así como la creación de un modelo predictivo que extraiga información valiosa a partir de estos datos.    

Así mismo, se plantea desplegar los modelos en un entorno de producción, permitiendo su uso en aplicaciones prácticas. 
De esta forma, cualquier usuario podrá suministrar mediciones de la variable que desee durante las últimas horas y obtener un pronóstico.

\section{Antecedentes y estado del arte}

El uso de redes neuronales y técnicas de deep learning (aprendizaje profundo) para la predicción de series temporales tiene sus orígenes en la década de 1940. Aunque el término deep learning ha ganado relevancia en los últimos años, los conceptos fundamentales y algunas de las técnicas más empleadas han existido desde hace mucho tiempo, evolucionando y perfeccionándose con el avance de la tecnología.
En 1943, Warren McCulloch y Walter Pitts fueron pioneros en desarrollar un modelo computacional para redes neuronales \cite{mcculloch1943}, 
estableciendo las bases de lo que hoy conocemos como inteligencia artificial basada en redes neuronales. 
Más adelante, en 1949, Donald Hebb introdujo la famosa Teoría Hebbiana, una hipótesis que proponía un mecanismo de aprendizaje basado en la plasticidad neuronal \cite{hebb1949}. 
Este principio se aplicó a modelos computacionales, impulsando la investigación en la simulación del aprendizaje humano.


En 1958, Frank Rosenblatt avanzó en este campo con la creación del perceptrón \cite{rosenblatt1958}, un algoritmo pionero de reconocimiento de patrones basado en una red de aprendizaje de computadora de dos capas.
Este modelo simple operaba mediante operaciones de adición y sustracción, sentando las bases para el desarrollo de redes neuronales más complejas.
Durante las siguientes décadas, la investigación en redes neuronales experimentó un estancamiento debido a limitaciones técnicas y a la falta de métodos eficaces para el entrenamiento de redes con múltiples capas. 
Sin embargo, en 1982, un gran avance revitalizó este campo: la introducción del algoritmo de propagación hacia atrás (backpropagation) \cite{werbos1982}, que resolvía el problema del entrenamiento eficiente de redes neuronales profundas, 
permitiendo la formación de redes multicapa de manera más rápida y eficaz.
Desde entonces, las redes neuronales han continuado evolucionando, impulsadas en gran parte por los avances en la potencia de cálculo de las GPU y el desarrollo del deep learning.
Hoy en día, esta tecnología es fundamental y revolucionaria, capaz de realizar predicciones de series temporales con notable rapidez y precisión.
La investigación en este ámbito sigue siendo prolífica. Una búsqueda rápida en internet revela la continua producción de proyectos y estudios enfocados 
en perfeccionar las técnicas de predicción temporal mediante aprendizaje profundo. Estos trabajos se caracterizan por un análisis exhaustivo y la exploración de nuevos métodos, 
todos con el objetivo de optimizar la precisión y la eficacia de las previsiones en aplicaciones prácticas.

\section{Objetivos}

\begin{enumerate}[label=\arabic*.] 
    \item Evaluar y seleccionar una fuente de datos.
    \item Evaluar y seleccionar el framework de aprendizaje profundo a usar en el proyecto.
    \item Diseño e implementación de la metodología de procesamiento de los datos.
    \item Evaluación y selección de arquitecturas pertinentes al problema.
    \item Implementación de una arquitectura adecuada.
    \item Evaluación del rendimiento de la solución propuesta.
    \item Comparación frente a metodologías tradicionales.
    \item Despliegue de la solución en un entorno de producción.
\end{enumerate}

\newpage
